\theme

{\setstretch{1.8}
\begin{center}
   \textsf{\huge SOMMAIRE}
\end{center}

% bandeau nombres et calculs
\begin{pspicture}(0,0)(\linewidth,1.5)
   \psframe*[linecolor=Crimson](0,0)(\linewidth,1)
   \rput(8.4,0.5){\textcolor{white}{\Large\textsf{NOMBRES ET CALCULS}}}
\end{pspicture}

\begin{multicols}{2}
   S01 Enchaînement d'opérations \pointilles \pageref{S01} \par
   S04 Les nombres relatifs \pointilles \pageref{S04} \par
   S07 Multiples et diviseurs \pointilles \pageref{S07} \par
   S10 Comparaison et égalité de fractions \pointilles \pageref{S10} \par
   S13 Expressions littérales \pointilles \pageref{S13} \par
   S16 Calculs avec des nombres relatifs \pointilles \pageref{S16} \par
   S19 Les nombres premiers \pointilles \pageref{S19} \par
   S22 Calculs avec des fractions \pointilles \pageref{S22} \par
   S25 Distributivité simple et égalités \pointilles \pageref{S25} \newline
\end{multicols}

\bigskip

% bandeau Espace et géométrie
\begin{pspicture}(0,0)(\linewidth,1.5)
   \psframe*[linecolor=DodgerBlue](0,0)(\linewidth,1)
   \rput(8.4,0.5){\textcolor{white}{\Large\textsf{GÉOMÉTRIE}}}
\end{pspicture}

\begin{multicols}{2}
   S02 Angles particuliers \pointilles \pageref{S02} \par
   S05 Repérage dans le plan \pointilles \pageref{S05} \par
   S08 Somme des angles d'un triangle \pointilles \pageref{S08} \par
   S11 La symétrie centrale \pointilles \pageref{S11} \par
   S14 Reconnaître des solides \pointilles \pageref{S14} \par      
   S17 L'inégalité triangulaire \pointilles \pageref{S17} \par
   S20 Le parallélogramme \pointilles \pageref{S20} \par
   S23 Représenter les solides \pointilles \pageref{S23} \par
   S26 Hauteurs et médiatrices du triangle \pointilles \pageref{S26} \par 
\end{multicols}

\bigskip

% bandeau organisation et gestion de données
\begin{pspicture}(0,0)(\linewidth,1.5)
   \psframe*[linecolor=DarkViolet](0,0)(\linewidth,1)
   \rput(8.4,0.5){\textcolor{white}{\Large\textsf{ORGANISATION ET GESTION DE DONNÉES}}}
\end{pspicture} 

\begin{multicols}{2}
   S06 Effectifs, fréquences et moyenne \pointilles \pageref{S06} \par
   S12 Notions de probabilités \pointilles \pageref{S12} \par
   S18 Le ratio \pointilles \pageref{S18} \par
   S24 La proportionnalité \pointilles \pageref{S22}
\end{multicols}

\bigskip

% bandeau grandeurs et mesures
\begin{pspicture}(0,0)(\linewidth,1.5)
   \psframe*[linecolor=ForestGreen](0,0)(\linewidth,1)
   \rput(8.4,0.5){\textcolor{white}{\Large\textsf{GRANDEURS ET MESURES}}}
\end{pspicture}

\begin{multicols}{2}
   S09 Horaires et durées \pointilles \pageref{S09} \par
   S15 Aires et périmètres \pointilles \pageref{S15} \par
   S21 Volume du pavé, du prisme et du cylindre \pointilles \pageref{S21} \par
   S27 L'aire du parallélogramme \pointilles \pageref{S27} \par
   S28 Propriétés des symétries \pointilles \pageref{S28} \par
\end{multicols}

\bigskip

% bandeau algorithmes et programmation
\begin{pspicture}(0,0)(\linewidth,1.5)
   \psframe*[linecolor=DarkOrange](0,0)(\linewidth,1)
   \rput(8.4,0.5){\textcolor{white}{\Large\textsf{ALGORITHMES ET PROGRAMMATION}}}
\end{pspicture}

\begin{multicols}{2}
   S03 En route vers la programmation \pointilles \pageref{S03}
\end{multicols}

\bigskip

Pour préparer les évaluations \pointilles \pageref{PDT} \par
%Corrigé des activités \pointilles \pageref{CDA}
}