\theme
\vspace*{1cm}

Ce manuel est composé de l'ensemble des activités, cours et exercices pour les classes de 5\up{e}D et 5\up{e}F du collège Simone Veil de Montpellier que j'ai à ma charge durant l'année 2023-2024. \par \smallskip
Il a été écrit en \LaTeX{} à l'aide notamment des classes \og ProfMaquette \fg{} et \og ProfCollege \fg{} de \href{https://ctan.org/author/poulain}{\bf Christophe Poulain}. \par
Une partie des exercices est issue du \og Manuel Sesamath, 5\up{e} \fg{} et du \og Cahier Sesamath, 5\up{e} \fg, éditions génération 5 et Magnard-Sesamath.

Si vous y voyez des erreurs ou des coquilles, même minimes, vous pouvez me les signaler à cette adresse : \href{mailto:nathalie.daval@ac-montpellier.fr}{nathalie.daval@ac-montpellier.fr}. Je remercie à ce propos {\it Jean-Félix Navarro} qui a effectué une relecture attentive du livret précédent, {\it Sébastien Lozano} pour son aide précieuse en \LaTeX{} et {\it Christophe Poulain} pour ses packages sans cesse mis à jour et les améliorations au fil de l'eau.  \par \bigskip

\bigskip

La progression est dite spiralée, c'est-à-dire que chaque \og chapitre \fg{} est décomposé en plusieurs séquences conçues pour durer une semaine en moyenne, ce qui permet de revoir les notions plusieurs fois dans l'année. La page suivante propose une programmation possible sur les cinq périodes (P1, P2, P3, P4 et P5) de l'année 2023-2024.

\bigskip

Chaque séquence du présent manuel est composée de la manière suivante : \par
\begin{itemize}
   \item \textcolor{red}{\bf Connaissances} et \textcolor{teal}{\bf compétences} : les connaissances et compétences évaluées au cycle 4 définies par le \href{https://eduscol.education.fr/document/621/download}{programme en vigueur à compter de la rentrée de l'année scolaire 2020}. \medskip
   \item \colorbox{Gold}{Débat} : un petit texte culturel illustré permettant d'échanger sur un thème en rapport au chapitre. Un morceau d'histoire, de l'étymologie, du vocabulaire, une curiosité mathématique\dots{} le tout agrémenté d'une courte vidéo de vulgarisation scientifique. \medskip
   \item \colorbox{SteelBlue}{\textcolor{white}{\sffamily\bfseries Activité d'approche}} : une activité à faire en classe, permettant de découvrir la notion de la séquence et de construire la trace écrite. \medskip
   \item \colorbox{SteelBlue!60!Black}{\textcolor{white}{\sffamily\bfseries Trace écrite}} : l'essentiel du cours à connaître et à maîtriser. \medskip
   \item \colorbox{DarkRed}{\textcolor{white}{\sffamily\bfseries Fiche d'exercices}} : une série d'exercices suivant les connaissances et compétences à acquérir ainsi que leur corrigé dans la version \og prof \fg. \par
      L'icône de plume \twemoji{feather} signifie que l'exercice se fait directement sur la fiche alors que tous les autres exercices doivent être faits sur le cahier. L'icône du petit monstre pixélisé \twemoji{alien monster} signifie quant à lui que l'exercice est un peu plus difficile que les autres. \medskip
   \item \colorbox{Salmon}{\textcolor{white}{\sffamily\bfseries Activité récréative}} : une activité ludique liée à la séquence, une énigme ou un problème à résoudre.
\end{itemize}

\bigskip

En fin de manuel se trouvent les {\sffamily VIDEOS DE TRAVAIL} données aux élèves, pour chaque notion. Elles sont composées de capsules issues sur site d'{\it Yvan Monka} : \href{https://www.maths-et-tiques.fr}{m@ths et tiques} et d'un line vers des exercices en ligne du site \textcolor{SteelBlue}{\sffamily {MathALÉA}} à faire pour réviser pour les évaluations.

\smallskip

Le corrigé des activités (d'approche et récréative) se trouve également en fin de manuel pour la version \og prof \fg{} de ce manuel.