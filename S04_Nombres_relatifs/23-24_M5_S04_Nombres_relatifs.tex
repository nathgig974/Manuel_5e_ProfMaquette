\themeN
\chapter{Les nombres relatifs}
\label{S04}

\programme%
   {\item Nombres décimaux négatifs, notion d'opposé.}
   {\item Comparer, ranger, encadrer des nombres en écriture décimale.
    \item Se repérer sur une droite graduée.}

\vfill

\begin{debat}{Débat : les unités de mesure de température}
   Il existe trois échelles principales de température :
   \begin{itemize}
      \item l'échelle Farenheit, créée en 1724 par le scientifique allemand {\bf Gabriel Farenheit} et dont la température de \Temp[F]{96} correspond à la température du corps humain ;
      \item l'échelle Celsius, créée en 1741 par le physicien suédois {\bf Anders Celsius}  dans laquelle \Temp{0} correspond au point de fusion de l'eau et \Temp{100} à son point d'ébullition ;
      \item l'échelle de Kelvin, créée à la fin du {\small XIX}\up{e} siècle par {\bf Lord Kelvin} pour laquelle le point \Temp[K]{0} correspond au zéro absolu, c'est-à-dire à la plus basse température existante.
   \end{itemize}
   \tcblower
      \psset{yunit=0.8}
      \begin{pspicture}(0,0)(8,7)
         \rput(1,0){\thermo}
         \rput[l](1.2,1){\Temp[F]{-459}}
         \rput[l](1.2,6.4){\Temp[F]{212}}
         \rput[l](1.2,4.7){\Temp[F]{32}}
         \rput(3.5,0){\thermo}
         \rput[l](3.7,1){\Temp{-273}} 
         \rput[l](3.7,4.7){\Temp{0}}
         \rput[l](3.7,6.4){\Temp{100}}
         \rput(6,0){\thermo}
         \rput[l](6.2,1){\Temp[K]{0}} 
         \rput[l](6.2,4.7){\Temp[K]{273}}  
         \rput[l](6.2,6.4){\Temp[K]{373}}         
        \end{pspicture}
\end{debat}

\hfill {\gray Vidéo : \href{https://www.youtube.com/watch?v=nzirDkQN99M}{\bf Celsius et Farenheit}, chaîne YouTube {\it Ma deuxième école}, épisode de la série {\it Culture G}.}


%%% Approche %%%
\begin{Maquette}[Cours]{Theme={Activité d'approche},Couleur={SteelBlue}}

   \AAtitre{Carrés magiques}

      {\it Objectifs : résoudre un problème avec des nombres ; montrer que, pour résoudre un problème, il est parfois nécessaire d’inventer de nouveaux nombres, des nombres.}

      \begin{AActivite}

         \AApartie{Les règles du jeu}
            \begin{minipage}{9cm}
               Un carré magique est un tableau carré tel que la somme pour chaque ligne, chaque colonne et chaque diagonale soit la même.
            \end{minipage}
            \qquad
            \begin{minipage}{6cm}
               \psset{unit=1.2}
               \begin{pspicture}(-0.5,-1.5)(4,3.25)
                  \psgrid[griddots=50, subgriddiv=0, gridlabels=0](0,0)(3,3)
                  \rput(0.5,0.5){4}
                  \rput(1.5,0.5){3}
                  \rput(2.5,0.5){8}
                  \rput(0.5,1.5){9}
                  \rput(1.5,1.5){5}
                  \rput(2.5,1.5){1}
                  \rput(0.5,2.5){2}
                  \rput(1.5,2.5){7}
                  \rput(2.5,2.5){6}
                  \rput(0.5,-0.3){$\downarrow$}
                  \rput(0.5,-0.7){15}
                  \rput(1.5,-0.3){$\downarrow$}
                  \rput(1.5,-0.7){15}
                  \rput(2.5,-0.3){$\downarrow$}
                  \rput(2.5,-0.7){15}
                  \rput(3.3,0.5){$\rightarrow$}
                  \rput(3.7,0.5){15}
                  \rput(3.3,1.5){$\rightarrow$}
                  \rput(3.7,1.5){15}
                  \rput(3.3,2.5){$\rightarrow$}
                  \rput(3.7,2.5){15}
                  \rput(3.3,-0.3){$\searrow$}
                  \rput(3.7,-0.7){15}
                  \rput(-0.3,-0.3){$\swarrow$}
                  \rput(-0.7,-0.7){15}
               \end{pspicture}
            \end{minipage}

         \AApartie{À vous de jouer !}
            Compléter les carrés suivants pour les rendre magiques en commençant par déterminer la somme commune.
            \begin{center}
            {\psset{unit=1.5,griddots=50, subgriddiv=0, gridlabels=0}
            \large
               \begin{pspicture}(0,-0.3)(4,4)
                  \psgrid(0,0)(3,3)
                  \rput(0.5,0.5){4}
                  \rput(0.5,2.5){8}
                  \rput(1.5,1.5){5}
                  \rput(2.5,0.5){2}
                  \rput(1.4,3.5){Somme = \pointilles[15mm]}
               \end{pspicture}
               \begin{pspicture}(-1,-0.3)(3,4)
                  \psgrid(0,0)(3,3)
                  \rput(2.5,2.5){24}
                  \rput(0.5,2.5){18}
                  \rput(1.5,1.5){15}
                  \rput(2.5,0.5){12}
                  \rput(1.4,3.5){Somme = \pointilles[15mm]}
               \end{pspicture}
            
               \begin{pspicture}(0,-0.3)(4,4)
                  \psgrid(0,0)(3,3)
                  \rput(1.5,2.5){7}
                  \rput(0.5,2.5){2}
                  \rput(1.5,1.5){3}
                  \rput(2.5,0.5){4}
                  \rput(1.4,3.5){Somme = \pointilles[15mm]}
               \end{pspicture}
               \begin{pspicture}(-1,-0.3)(3,4)
                  \psgrid(0,0)(3,3)
                  \rput(2.5,2.5){4}
                  \rput(0.5,0.5){10}
                  \rput(1.5,1.5){7}
                  \rput(1.5,2.5){1}
                  \rput(1.4,3.5){Somme = \pointilles[15mm]}
               \end{pspicture}}
            \end{center}  

   \end{AActivite}
   
   \vfill\hfill{\footnotesize\it Source : \og Une introduction des nombres relatifs en 5\up{e} \fg, PLOT 45, APMEP 2014.}

\end{Maquette}


%%% Trace écrite %%%
\begin{Maquette}[Cours]{Theme={Trace écrite},Couleur={0.4[SteelBlue,Black]}}

   %%% 1
   \section{Nombres relatifs}

      \begin{definition*}{}
         Un {\bf nombre relatif} est un nombre positif ($+$) ou négatif ($-$). Le nombre sans son signe correspond à sa distance à l'origine 0.
      \end{definition*}

      \begin{exemple*}{}
         Les étages d'un immeuble sont  repérés par rapport à un niveau 0 : le rez-de-chaussée. Les étages au-dessus sont les étages positifs et les étages en dessous (cave, garages) sont les étages négatifs.
      \end{exemple*}

      \begin{exemple*}{}
         Le signe de $+3$ est $+$ et sa distance à l'origine 0 est 3. \par
         Le signe de $-7,5$ est $-$ et sa distance à l'origine 0 est 7,5.  
      \end{exemple*}

      \begin{definition*}{}
         L'{\bf opposé} d'un nombre relatif est le nombre de signe contraire et de même	
      distance à 0.
      \end{definition*}

      \begin{exemple*}{}
         L'opposé de $-3$ est $+3$ et l'opposé de $+2,68$ est $-2,68$. \par
         L'opposé de 0 rest 0.
      \end{exemple*}

      De manière usuelle, on omet le signe \og $+$ \fg{} devant les nombres positifs.


   %%% 2
   \section{Droite graduée et comparaison}

      \begin{definition*}{}
         Sur une droite graduée, on repère chaque point par un nombre : son abscisse. \par
         D'un côté de l'origine 0, on place les nombres négatifs et de l'autre les nombres positifs.
         \begin{center}
            \begin{pspicture}(-5,-0.5)(5,0.8)
               \psaxes[yAxis=false]{->}(0,0)(-5,0)(5,0)
               \psline[linecolor=Crimson]{<->}(-5,0.3)(0,0.3)
               \rput(-2.5,0.6){\textcolor{Crimson}{nombres négatifs}}
               \psline[linecolor=DodgerBlue]{<->}(0,0.3)(5,0.3)
               \rput(2.5,0.6){\textcolor{DodgerBlue}{nombres positifs}}
            \end{pspicture}
         \end{center}
      \end{definition*}

      \begin{exemple*}{}
         L'abscisse de A est $-3$, on note A($-3$) ; l'abscisse de B est 0, l'abscisse de C est $+4$.
         \begin{center}
            %\Reperage[AffichageGrad,AffichageNom,AffichageAbs=1]{-5/,4.5/C,-3/A,0/B,5/}
            \begin{pspicture}(-5,-1)(5,-0.6)
               \psline[linecolor=Crimson]{<->}(-3,-0.9)(0,-0.9)
               \rput(-1.5,-1.2){\textcolor{Crimson}{\small distance à l'origine : 3}}
               \psline[linecolor=DodgerBlue]{<->}(0,-0.9)(4.5,-0.9)
               \rput(2.25,-1.2){\textcolor{DodgerBlue}{\small distance à l'origine : 4,5}}
            \end{pspicture}
         \end{center}
      \end{exemple*}

      \begin{propriete*}{}
         \begin{itemize}
            \item Deux nombres relatifs négatifs sont rangés dans l'ordre inverse de leur
      distance à zéro.
            \item Un nombre relatif négatif est inférieur à un nombre relatif positif.
            \item Deux nombres relatifs positifs sont rangés dans l'ordre de leur distance à zéro.
         \end{itemize}  
      \end{propriete*}

      \begin{exemple*}{}
         $-4<-2$ car $4>2$ \qquad ; \quad $-4<2$ car $-4<0$ et $2>0$ \qquad ; \quad $+4>+2$ car $4>2$.
      \end{exemple*}

      Les nombres négatifs sont rangés \og dans le sens inverse \fg{} des nombres positifs.

\end{Maquette}


%%% Exercices %%%
\begin{Maquette}[Fiche,CorrigeFin,Colonnes=2]{}
   
   \begin{multicols}{2}

      \begin{exercice}[SLF] %1
         Quelle est la température indiquée par chacun des thermomètres ? \par
         \hspace*{-5mm}
         \Reperage[Thermometre,Pasx=10,ValeurUnitex=5,Mercure]{4/A,16/,16/}
         \Reperage[Thermometre,Pasx=10,ValeurUnitex=10,Mercure]{-12/A,-15/,17/}
         \Reperage[Thermometre,Pasx=5,Mercure]{-2/A,-10/,6/}
         \Reperage[Thermometre,Pasx=5,ValeurUnitex=0.5,Mercure]{7/A,-5/,11/}
      \end{exercice}

      \begin{Solution}
         On peut lire les températures suivantes : \par
         \cor{\Temp{2} \hfill \Temp{-12} \hfill \Temp{-0,4} \hfill \Temp{0,7} \hfill}
      \end{Solution}
      
      
      \begin{exercice}[SLF] %2
         Colorier les thermomètres jusqu'à la graduation correspondant à la température donnée. \par
         \Reperage[Thermometre,Pasx=10,ValeurUnitex=10]{-16/,15/}
         \Reperage[Thermometre,Pasx=5,ValeurUnitex=1]{-10/,6/}
         \Reperage[Thermometre,Pasx=10,ValeurUnitex=1]{-20/,12/}
         \Reperage[Thermometre,Pasx=10,ValeurUnitex=5]{-15/,17/} \par
         \hskip11mm $-11$ \hskip12mm $-1,2$ \hskip12mm $-0,5$ \hskip15mm 6
      \end{exercice}

      \begin{Solution}
         On a les hauteurs de mercure suivantes : \par
         \Reperage[Thermometre,Pasx=10,ValeurUnitex=10,Mercure,CouleurMercure=RoyalBlue]{-11/A,-16/,15/}
         \Reperage[Thermometre,Pasx=5,ValeurUnitex=1,Mercure,CouleurMercure=RoyalBlue]{-6/A,-10/,6/}
         \Reperage[Thermometre,Pasx=10,ValeurUnitex=1,Mercure,CouleurMercure=RoyalBlue]{-5/A,-20/,12/}
         \Reperage[Thermometre,Pasx=10,ValeurUnitex=5,Mercure,CouleurMercure=RoyalBlue]{12/A,-15/,17/}
      \end{Solution}
      
      
      \begin{exercice}[SLF] %3
         Entourer en bleu les nombres positifs et en rouge les nombres négatifs.
         \begin{center}
               {\hautab{1.8}
               \begin{tabular}{|*{5}{C{1}|}}
               \hline
               12 & $+\pi$ & $-\dfrac{12}{13}$ & $-17$ & 0,001 \\
               \hline
               $-54,2$ & $\dfrac{1}{10}$ & $-0,14$ & $\dfrac{3}{7}$ & 100,01 \\
               \hline
               12,6 & $-1,18$ & $-3^2$ & $+0,1$ & 0 \\
               \hline
            \end{tabular}}
         \end{center}
      \end{exercice}

      \begin{Solution}
         \medskip
         {\hautab{1.8}
         \begin{tabular}{|*{5}{C{1.05}|}}
            \hline
            \fcolorbox{RoyalBlue}{white}{12} & \fcolorbox{RoyalBlue}{white}{$+\pi$} & \fcolorbox{red}{white}{$-\dfrac{12}{13}$} & \fcolorbox{red}{white}{$-17$} & \fcolorbox{RoyalBlue}{white}{$0,001$} \\
            \hline
            \fcolorbox{red}{white}{$-54,2$} & \fcolorbox{RoyalBlue}{white}{$\dfrac{1}{10}$} & \fcolorbox{red}{white}{$-0,14$} & \fcolorbox{RoyalBlue}{white}{$\dfrac{3}{7}$} & \fcolorbox{RoyalBlue}{white}{100,01} \\
            \hline
            \fcolorbox{RoyalBlue}{white}{12,6} & \fcolorbox{red}{white}{$-1,18$} & \fcolorbox{red}{white}{$-3^2$} & \fcolorbox{RoyalBlue}{white}{0,1} & \fcolorbox{red}{white}{\fcolorbox{RoyalBlue}{white}{0}} \\
            \hline
         \end{tabular}}
      \end{Solution}
      
      
      \begin{exercice}[SLF] %4
         Compléter le tableau suivant :
         \begin{center}
            {\hautab{1.3}
            \begin{tabular}{|*{7}{c|}}
               \hline
               Nombre & 2,5 & & 0 & $-5$ & & 7,1 \\
               \hline
               Opposé & & \!\!$-2,7$ & & & 1 & \\
               \hline 
            \end{tabular}}
         \end{center}
      \end{exercice}

      \begin{Solution}
         \medskip
         {\hautab{1.3}
         \begin{tabular}{|*{7}{c|}}
            \hline
            Nombre & 2,5 & \textcolor{RoyalBlue}{2,7} & 0 & $-5$ & \textcolor{RoyalBlue}{$-1$} & 7,1 \\
            \hline
            Opposé & \!\textcolor{RoyalBlue}{$-2,5$} & \!$-2,7$ & \textcolor{RoyalBlue}{0} & \textcolor{blue}{5} & \; 1 & \!\!\textcolor{RoyalBlue}{$-7,1$} \\
            \hline 
         \end{tabular}}
      \end{Solution}
      

      \begin{exercice} %5
         Reproduire l'axe chronologique ci-dessous puis placer le plus précisément possible ces évènements : \par \medskip
         \Reperage[Unitex=0.75,ValeurUnitex=100]{4/B,-5/A}
         \begin{itemize}
            \item T : temple de Jérusalem est détruit en 70 après J.-C.
            \item J : Jules César naît en 100 avant J.-C.
            \item C : Constantin crée Constantinople en 324
            \item A : Alexandre le Grand meurt en 324 avant J.-C.
         \end{itemize}
      \end{exercice}

      \begin{Solution}
         Droite graduée complétée : \par
         \Reperage[Unitex=0.75,ValeurUnitex=100,AffichageNom,AffichageAbs=1]{-5/,-1/J,0.7/T,3.24/C,-3.24/A,4/}
      \end{Solution}
      
      
      \begin{exercice} %6 
         Construire une droite graduée dont l'origine est au milieu du cahier et l'unité vaut \Lg{1} puis répondre aux questions suivantes.
         \begin{enumerate}
            \item Sur la droite graduée, placer les points : \par
               A($+8$), B($-2$), C($+3$), D($-5$) et E($+2$).
            \item En examinant la position des points A, B, C, D et E sur cette droite graduée, comparer : \par
               \begin{multicols}{3}
                  $+2$ et $-2$ \par
                  $-2$ et $-5$ \par
                  $+2$ et $-5$ \par
                  $+8$ et $-2$ \par
                  $+3$ et $+8$ \par
                  $-5$ et $+3$
               \end{multicols}
            \item Ranger dans l'ordre croissant : $+8 ; -2 ; 3 ; -5 ; +2$.
         \end{enumerate}
      \end{exercice}

      \begin{Solution}
         \begin{enumerate}
            \item Droite graduée complétée (à l'échelle 1/2) : \par
            \hspace*{-10mm} %\Reperage[Unitex=0.5,AffichageNom,AffichageAbs=1]{-6/,-5/D,-2/B,2/E,3/C,8/A,9/}
            \item On peut comparer directement par lecture graphique :
               \begin{multicols}{3}
                  $+2 \; \cor{>} \, -2$ \par
                  $-2  \; \cor{>} \, -5$ \par
                  $+2 \; \cor{>} \, -5$ \par
                  $+8 \; \cor{>} \, -2$ \par
                  $+3 \; \cor{<} \, +8$ \par
                  $-5 \; \cor{<} \, +3$
               \end{multicols}
            \item \cor{$-5<-2<+2<+3<+8$}.
         \end{enumerate}
      \end{Solution}
      
      
      \begin{exercice}[SLF] %7
         Compléter par <, > ou =.
         {\baselineskip=7mm
         \begin{colenumerate}
            \item $+5,34 \pointilles +3,54$
            \item $0,05 \pointilles 1$
            \item $-8,51 \pointilles -8,5$
            \item $11,9 \pointilles +11,9$
            \item $3,14 \pointilles -1,732$
            \item $-9,27 \pointilles -9,272$
            \item $+8,64 \pointilles -8,64$
            \item $-19,2 \pointilles +9,2$
            \item $-14,39 \pointilles +14,4$
            \item $-0,99 \pointilles -0,909$
         \end{colenumerate}}
      \end{exercice}

      \begin{Solution}
         {\baselineskip=7mm
         \begin{colenumerate}
            \item $+5,34 \; \cor{>} \, +3,54$
            \item $0,05 \; \cor{<} \, 1$
            \item $-8,51 \; \cor{<} \, -8,5$
            \item $11,9 \; \cor{=} \, +11,9$
            \item $3,14 \; \cor{>} \, -1,732$
            \item $-9,27 \; \cor{>} \, -9,272$
            \item $+8,64 \; \cor{>} \, -8,64$
            \item $-19,2 \; \cor{<} \, +9,2$
            \item $-14,39 \; \cor{<} \, +14,4$
            \item $-0,99 \; \cor{<} \, -0,909$
         \end{colenumerate}}
      \end{Solution}
      

      \begin{exercice} %8
         Ranger dans l'ordre croissant et simplifier.
         {\baselineskip=7mm
         \begin{enumerate}
            \item $+3 \; ; \; -7 \;;\;-8 \;;\; +7 \;; \;+14\; ;\; +8 \;;\; -9$
            \item $+5,0\; ; \;+2,7 \;;\; -2,6\; ; \;-3,1\; ; \;+7,1\; ; \;-8,3\; ;\; -0,2$
            \item $-10,6 \; ; +14,52\; ; -8,31 \;; -3,8 \;; +4,2 \;; +14,6\; ; -8,3$
         \end{enumerate}}  
      \end{exercice}

      \begin{Solution}
         {\baselineskip=7mm
         \begin{enumerate}
            \item \cor{$-9<-8<-7<+3<+7<+8<+14$.}
            \item \cor{$-8,3<-3,1<-2,6<-0,2<2,7<5<7,1$.}
            \item \cor{$-10,6<-8,31<-8,3<-3,8<4,2<14,52$.}
            \hfill $<14,6$.
         \end{enumerate}}  
      \end{Solution}
      
      
      \begin{exercice}[Dur] %9
         Chasser l'intrus dans chacun des cas suivants.
         {\baselineskip=7mm
         \begin{enumerate}
            \item $-9,84 < -9,72 < -9,67 < -9,78 < -9,18$
            \item $+1,5 < +1,51 < +1,499 < +1,54 < +1,55$
            \item $-1\,002 > -1\,220 > -1\,022 > -1\,202 > -1\,222$
         \end{enumerate}}
      \end{exercice}

      \begin{Solution}
         {\baselineskip=7mm
         \begin{enumerate}
            \item $-9,84 < -9,72 < -9,67 < \cor{\cancel{-9,78}} < -9,18$ 
            \item $+1,5 < +1,51 < \cor{\cancel{+1,499}} < +1,54 < +1,55$ 
            \item $-1\,002 > \cor{\cancel{-1\,220}} > -1\,022 > -1\,202 > -1\,222$
         \end{enumerate}}
      \end{Solution}
      
      
      \begin{exercice}  %10
         Donner tous les entiers relatifs compris entre :
         \begin{colenumerate}
            \item $-2,3$ et $+5,7$.
            \item $-20$ et $-14,8$.
         \end{colenumerate}
      \end{exercice}
      
      \begin{Solution}
         \begin{enumerate}
            \item Entre $-2$ et $+5$ : \cor{$-1 \; ; \; 0 \; ; \; 1 \; ; \; 2 \; ; \; 3 \; ; \; 4$}.
            \item Entre $-15$ et $-20$ : \cor{$-19 \; ; \; -18 \; ; \; -17 \; ; \; -16$}.
         \end{enumerate}
      \end{Solution}
               
   \end{multicols}

\end{Maquette}


%%% Récré %%%
\begin{Maquette}[Cours]{Theme={Activité récréative},Couleur={IndianRed}}
    
   \ARtitre{Nombres croisés}

      Compléter cette grille de nombres croisés à l'aide de chiffres et de signes \og $+$ \fg{} ou \og $-$ \fg{} grâce aux indications données. \par \medskip

      \begin{center}
         {\hautab{2.5}
         \begin{tabular}{|>{\columncolor[gray]{.8}}C{0.84}|*{6}{C{0.84}|}}
            \hline
            \rowcolor{gray!40} & a & b & c & d & e & f \\
            \hline
            1 & & \cellcolor{black!70} & & & \cellcolor{black!70} & \\
            \hline
            2 & & & & \cellcolor{black!70} & & \\
            \hline
            3 & & \cellcolor{black!70} & \cellcolor{black!70} & & & \cellcolor{black!70} \\
            \hline
            4 & \cellcolor{black!70} & & & \cellcolor{black!70} & \cellcolor{black!70} & \\
            \hline
            5 & & & \cellcolor{black!70} & & & \\
            \hline
            6 & & \cellcolor{black!70} & & & & \cellcolor{black!70} \\
            \hline
         \end{tabular}}
      \end{center}

      \vskip1cm
      \begin{multicols}{2}
      {\bf Horizontalement} \par
      \begin{enumerate}
         \item Valeur du plus grand chiffre. \par
            Opposé de l'entier compris entre $-12,2$ et $-13,9$. \par
            Les nombres négatifs sont précédés de ce signe. \par
         \item Résultat du calcul $8\times20-(12+28)$. \par
            Nombre entier compris entre $-1,8$ et $-0,2$. \par
         \item Opposé de l’opposé de $+8$. \par
            Nombre entier supérieur à 73,01 et inférieur 74,99. \par
         \item Sur une droite graduée de 3 en 3, je suis placé à trois graduations à gauche de l'origine. \\
            Signe de l’opposé d'un nombre positif. \par
         \item Nombre entier le plus proche $-1,4$. \par
            Nombre entier inférieur à $-15,154$ et supérieur à $-16,98$. \par
         \item Diviseur commun à 12 ; 24 et 33. \par
            Mon chiffre des centaines est le double de mon chiffre des dizaines qui est lui-même le double de mon chiffre des unités. \par
         \end{enumerate}
      \columnbreak
      {\bf Verticalement} \par
      \begin{enumerate}
         \item[a.] Résultat du calcul $9\times(100+2)$. \par
            Nombre relatif inférieur à zéro et se trouvant à 5 unités du nombre $+2$. \par
         \item[b.] J'ai la même distance à zéro que le nombre $-2$. \par
            Nombre opposé de la moitié de 2. \par
         \item[c.] Le chiffre des unités est l'abscisse de l'origine et le chiffre des dizaines est le premier nombre entier positif non nul. \par
            Opposé de l'entier compris entre $-9,12$ et $-8,93$. \par
            Nombre relatif se situant après zéro et se trouvant à $11$ unités du nombre $-7$. \par
         \item[d.] Distance à zéro de l'opposé de $-\dfrac{33}{11}$. \par
            Opposé de $-42\div6$. \par
            Nombre négatif se trouvant à deux unités de l'origine. \par
         \item[e.] Nombre se trouvant à 8 unités de $-12$. \par
            Distance à zéro de $+\dfrac{22}{2}$. \par
         \item[f.] Opposé de $+1$. \par
            Nombre entier le plus proche et supérieur à $-6,98$.
      \end{enumerate}

   \end{multicols}

\end{Maquette}