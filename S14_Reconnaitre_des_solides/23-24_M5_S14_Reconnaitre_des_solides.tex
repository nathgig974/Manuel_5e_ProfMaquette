\graphicspath{{../../S14_Reconnaitre_des_solides/Images/}}

\themeG
\chapter{Reconnaître des solides}
\label{S14}

\textcolor{red}{\bf Connaissances :}
   \begin{connaissances}
      \item Reconnaître des solides : pavé droit, cube, prisme, cylindre, pyramide, cône et boule.
   \end{connaissances}

\vfill

\begin{debat}{Débat : les solides de Platon}
   Parmi les solides de l'espace, il en est une sorte qui a été étudiée par le philosophe grec {\bf Platon} ($-425;-348$) : les polyèdres réguliers et convexes. Ce dernier associe chacun des quatre éléments physiques avec un solide régulier.
   \begin{itemize}
      \item la {\bf Terre} est associée au {\it cube} : ces petits solides font de la poussière lorsqu'ils sont émiettés et se cassent lorsqu'on s'en saisit ;
      \item l'{\bf air} est associé à l'{\it octaèdre} : ses composants minuscules sont si doux qu'on peut à peine les sentir ;
      \item l'{\bf eau} est associée à l'{\it icosaèdre} : elle s'échappe de la main lorsqu'on la saisit comme si elle était constituée de petites boules minuscules ;
      \item le {\bf feu} est associé au {\it tétraèdre} car la chaleur du feu semble pointue comme un poignard ;
      \item le \textit{dodécaèdre} est mis en correspondance avec le {\bf tout}, parce que c'est le solide qui ressemble le plus à la sphère.
   \end{itemize}
   \tcblower
      \begin{pspicture}(-1,-1)(15,1.8)
         \psset{faceName=\arabic,Frame=false,faceNameFont=\sffamily}
         \rput(0,-0.2){\psTetrahedron}
         \rput(3,0.3){\psHexahedron[psscale=0.8]}
         \rput(6,0.2){\psOctahedron[psscale=1.5,faceNameFont=\scriptsize]}
         \rput(8.8,0.3){\psDodecahedron[psscale=0.8]}
         \rput(11.6,0.2){\psIcosahedron[psscale=0.7]}
      \end{pspicture}
\end{debat}

\hfill {\gray Vidéo : \href{ttps://www.yout-ube.com/watch?v=eDsFmYur9Yo}{\bf Les 5 solides de Platon}, chaîne YouTube {\it Micmaths} de {\it Mickaël Launay}.}


%%% Approche %%%
\begin{Maquette}[Cours]{Theme={Activité d'approche},Couleur={SteelBlue}}

   \AAtitre{Les polydrons}

      {\it Objectifs : construire des solides fermés ; trier des solides selon leur forme.}

      \begin{AActivite}

         Les Polydrons sont des polygones en plastique dur qui peuvent se fixer entre eux à l'aide de charnières. \par
         Ce matériel permet de construire facilement des polyèdres et des patrons. \par
         
         \AApartie{Construction de solides}
            \begin{enumerate}
               \item Citer les différentes formes de Polydrons en précisant leur nature exacte. \par \medskip
                  \pointilles
               \item Construire un premier solide, donner son nom si possible et le dessiner. \par \vskip35mm      
               \item Construire d'autres solides en essayant de varier les formes.
            \end{enumerate} 

         \AApartie{Classement des solides}
            \begin{enumerate}
               \item En regroupant tous les solides de la classe, déterminer un classement commun, discuter des choix.
               \item Citer les classes choisies en expliquant leurs caractéristiques. \par \medskip
                  \pointilles\par \medskip
                  \pointilles \par \medskip
                  \pointilles \par \medskip
                  \pointilles \par \medskip
                  \pointilles \par \medskip
            \end{enumerate}
            \begin{center}
               \includegraphics[width=16cm]{polydrons}
            \end{center}

   \end{AActivite}

\end{Maquette}


%%%Trace écrite %%%
\begin{Maquette}[Cours]{Theme={Trace écrite},Couleur={0.4[SteelBlue,Black]}}

   \begin{pspicture}(-0.5,0)(17,24)
      \rput(9.25,10){\ovalnode{A}{\Large Les solides du collège}}
      \rput{90}(0,5){\textcolor{Crimson}{\large Les solides non polyédriques}}
      \rput{90}(0,15){\textcolor{DodgerBlue}{\large Les polyèdres}}
      %prisme
      \psnode(6,12.5){B}{\begin{minipage}{10.5cm}{\bf Prisme} : du grec {\it prismatos}, scié. Deux bases polygonales, des faces latérales qui sont des parallélogrammes, rectangles si le prisme est droit \end{minipage}}
      \rput(5,13.5){{\psset{unit=0.35}\psSolid[object=prisme,h=0.8,action=draw*,linecolor=DodgerBlue]}}
      \rput(8.5,14){\includegraphics[width=3.25cm]{Cours_Toblerone}}
      \ncline{A}{B}
      \ncput*{\textcolor{DodgerBlue}{prismes}}
      %pavé 
      \psnode(6,16){D}{\begin{minipage}{10.5cm}{\bf Parallélépipède ou pavé} : du grec {\it parallelos}, parallèle et {\it epidon}, surface. Cas particulier du prisme droit lorsque la base est un rectangle \end{minipage}}
      \rput(4.5,17.5){\psSolid[object=parallelepiped,a=0.6,b=0.4,c=0.3,action=draw*,linecolor=DodgerBlue]}
      \rput(8.5,17.5){\includegraphics[width=2.5cm]{Cours_boite}}
      \psnode(6,13){G}{} 
      %\ncline{G}{D}
      %cube
      \psnode(6,19.5){E}{\begin{minipage}{10.5cm}{\bf Cube} : cas particulier du pavé droit lorsque toutes les faces sont carrées \end{minipage}}
      \rput(4.5,21){\psSolid[object=parallelepiped,a=0.5,action=draw*,RotX=30,linecolor=DodgerBlue]}
      \rput(8.5,21){\includegraphics[width=2cm]{Cours_Rubiks}}
      \psnode(6,17){F}{}
      %\ncline{F}{E}
      %Pyramide
      \psnode(14.75,14){C}{\begin{minipage}{4.5cm}{\bf Pyramide} : une base polygonale, un sommet, des faces latérales triangulaires, qui sont isocèles et superposables si la pyramide est régulière \end{minipage}}
      \rput(15,17){\psSolid[object=tetrahedron,r=0.6,action=draw*,RotZ=70,linecolor=DodgerBlue]}
      \rput(14.75,20.5){\includegraphics[width=4.5cm]{Cours_Kheops}}
      \ncline{A}{C}
      \ncput*{\textcolor{DodgerBlue}{pyramides}}
      %Cylindre
      \psnode(3.75,6.5){D}{\begin{minipage}{4.5cm}{\bf Cylindre} : du grec {\it kulindros}, rouleau. Deux bases en forme de disques, une surface latérale \end{minipage}}
      \rput(4.5,3.5){\psSolid[object=cylindre,h=1,r=0.2,action=draw**,ngrid=8 16,RotX=90,linecolor=Crimson]}
      \rput(3.75,1){\includegraphics[height=2.75cm]{Cours_conserve}}
      \ncline{A}{D}
      \ncput*{\textcolor{Crimson}{cylindres}}
      %Cône
      \psnode(9.25,6.5){D}{\begin{minipage}{4.5cm}{\bf Cône} : du grec {\it kônos}, pomme de pain. Une base en forme de disque, une surface latérale, un sommet \end{minipage}}
      \rput(9.25,4.85){\psSolid[object=cone,h=0.8,r=0.4,action=draw**,ngrid=8 16,RotX=200,linecolor=Crimson]}
      \rput(10,1){\includegraphics[height=3cm]{Cours_glace}}
      \ncline{A}{D}
      \ncput*{\textcolor{Crimson}{cônes}}
      %Sphère
      \psnode(14.75,6.5){D}{\begin{minipage}{4.5cm}{La \bf sphère} : du grec {\it sphaîra}, corps rond, est la surface extérieure de la {\bf boule} \end{minipage}}
      \rput(14.75,4.15){\psSolid[object=sphere,r=0.45,ngrid=18 18,linecolor=Crimson]}
      \rput(14.75,1){\includegraphics[height=2.75cm]{Cours_ballon}}
      \ncline{A}{D}
      \ncput*{\textcolor{Crimson}{boules}}
   \end{pspicture}

\end{Maquette}


%%% Exercices %%%
\begin{Maquette}[Fiche,CorrigeFin,Colonnes=2]{}

   \begin{multicols}{2}

      \begin{exercice} %1
         On considère les catégories de solides suivants que l'on peut trouver dans la vie courante :
         \begin{enumerate}
            \item[A.] Différentes boites parallélépipédiques et cubiques.
            \item[B.] Des prismes droits à bases triangulaires (Toblerone) ou octogonales.
            \item[C.] Une pyramide à base carrée tronquée (boite de fromage de chèvre).
            \item[D.] Des cylindres (boîtes de conserve diverses, boîte de camembert, rouleau de papier d'aluminium).
            \item[E.] Des cônes.
            \item[F.] Des boules (balles, ballons).
            \item[G.] D'autres emballages (formes ovales, anneaux, boîtes en forme de c\oe ur\dots).
         \end{enumerate}
         \begin{enumerate}
            \item Classer ces sept catégories selon leur caractère polyédrique ou non
            \item Pourquoi, en cours de maths, faut-il s'intéresser uniquement aux formes des boites d'emballages ?
            \item Quelle est la particularité du rouleau de papier d'aluminium ?
            \item Des classes ont proposé les classements suivants :
               \begin{center}
                  {\hautab{1}
                  \begin{tabular}{|*{2}{C{2}|}}
                     \hline
                     5\up{e}1 & 5\up{e}2 \\
                     \hline
                     A, B et C & A, B et D \\
                     F & C et E \\
                     G & F et G \\
                     \hline
                  \end{tabular}}
               \end{center}
               Quels ont pu être leurs critères de classement ?
            \item On propose la définition suivante pour déterminer un polyèdre : \og solide qui ne roule pas \fg{} et pour un non polyèdre : \og solide qui roule \fg. \par
               Ces définition vous paraissent-elles pertinentes ?
         \end{enumerate}
      \end{exercice}
         
      \begin{Solution}
         \begin{enumerate}
            \item On obtient le classement suivant :
            \begin{itemize}
               \item \cor{les polyèdres : A, B, C ;}
               \item \cor{les non polyèdres : D, E, F, G}. 
            \end{itemize}
            \item On pourrait proposer un classement du type \og sucré, salé \fg{}, ou \og ça se mange - ça ne se mange pas \fg{} ou encore proposer un classement par couleur, par taille\dots{} ce qui n'est pas l'objectif attendu.
            \item Le rouleau de papier d'aluminium est un \cor{solide non fermé}. 
            \item 
               \begin{itemize}
                  \item Pour les 5\up{e}1, il semble que les élèves ont \og repéré \fg{} les polyèdres A, B et C. \par
                     Les boules forment une deuxième catégorie. \par
                     Les autres emballages une troisième. \par
                     On remarque que les cylindres et les cônes n'apparaissent pas dans le classement, sûrement en raison de leur forme mélangeant des faces planes comme pour les polyèdres A, B, C et des surfaces non planes comme pour les boules F.
                  \item Pour les 5\up{e}2, les élèves ont classé ensemble les prismes et les cylindres, c'est-à-dire les solides ayant des faces opposées parallèles. \par
                     Puis ils ont mis ensemble les solides \og pointus \fg{} comme les pyramides et les cônes. \par
                     Enfin, les boules et autres emballages, constituent la catégorie des formes \og arrondies \fg{}, les cylindres et les cônes ayant déjà été classés. 
               \end{itemize}
            \item Cette définition n'est pas pertinente : par exemple un cône peut rouler ou pas selon si on le pose sur sa base ou sur le côté. \par
            Un polyèdre régulier avec de multiples faces donne l'impression qu'il roule lorsqu'on le lance.
         \end{enumerate}
      \end{Solution}
      
      
      \begin{exercice}[SLF] %2
         On dispose de trois objets sur une table : un cône, un cube et une sphère. On a également des représentations de ces solides selon des points de vue différents. \\ [2mm]
         {\bf Vue de la table du dessus :} 
         \begin{center}
            \begin{pspicture}(-2.3,-2.1)(2.3,2.3)
               \psframe(-1.5,-1.5)(1.5,1.5)
               \rput(2,2){NE}
               \rput(1.7,1.7){$\swarrow$}
               \rput(2,0){E}
               \rput(1.7,0){$\leftarrow$}
               \rput(2,-2){SE}
               \rput(1.7,-1.7){$\nwarrow$}
               \rput(0,-2){S}
               \rput(0,-1.7){$\uparrow$}
               \rput(-2,-2){SO}
               \rput(-1.7,-1.7){$\nearrow$}
               \rput(-2,0){O}
               \rput(-1.7,0){$\rightarrow$}
               \rput(-2,2){NO}
               \rput(-1.7,1.7){$\searrow$}
               \rput(0,2){N}
               \rput(0,1.7){$\downarrow$}
               \pscircle(0.6,-0.85){0.45}
               \pscircle(-0.25,0.85){0.45}
               \pscircle(-0.25,0.85){0.05}
               \pspolygon(-1,-0.4)(-0.55,0.35)(0.2,-0.1)(-0.25,-0.85)
            \end{pspicture}
         \end{center}
         {\bf Solides en vue du dessus et de face :}
         \begin{center}
            \begin{pspicture}(0,-1.5)(7.9,1.9)
               \rput[l](0.5,1){vue aérienne}
               \pscircle(3.5,1){0.45}
               \pscircle(3.5,1){0.05}
               \pscircle(5.25,1){0.45}
               \rput(7.4,1.25){\pspolygon(-1,-0.4)(-0.55,0.35)(0.2,-0.1)(-0.25,-0.85)}
               \rput[l](0.5,-0.5){vue frontale}
               \rput(3.05,-1.3){\cone}
               \rput(5.25,-.95){\boule}
               \rput(6.4,-0.95){\cube}
            \end{pspicture}
         \end{center} 
         Les images ci-dessous représentent des vues, selon divers axes de visée. Déterminer le point de vue de chaque image (attention, certaines images correspondent à aucune configuration !). \\ [2mm]
         {\hautab{1.5}
         \begin{tabular}{|*{8}{C{0.55}|}}
            \hline
            N & NE & E & SE & S & SO & O & NO \\
            \hline
            & & & & & & & \\
            \hline
         \end{tabular}} \smallskip
      \end{exercice} 
      
      \begin{Solution}
         Les images D et H ne sont pas utilisées. \par \smallskip
         {\small\hautab{1.5}
         \begin{tabular}{|*{8}{C{0.55}|}}
            \hline
            N & NE & E & SE & S & SO & O & NO \\
            \hline
            \textcolor{RoyalBlue}{A} & \textcolor{RoyalBlue}{J} & \textcolor{RoyalBlue}{E} & \textcolor{RoyalBlue}{I} & \textcolor{RoyalBlue}{C} & \textcolor{RoyalBlue}{G} & \textcolor{RoyalBlue}{B} & \textcolor{RoyalBlue}{F} \\
            \hline
         \end{tabular}}
      \end{Solution} 
         
   \end{multicols}
         
   \begin{center}
      \begin{pspicture}(0,0)(3.3,2.9)
         \psframe(0,0)(3.2,2.5)
         \rput(2.9,2.2){A}
         \rput(0.9,0.2){\boule}
         \rput(1.35,0.2){\cube}
         \rput(1.35,0.2){\cone}
      \end{pspicture}
      \begin{pspicture}(0,0)(3.3,2.9)
         \psframe(0,0)(3.2,2.5)
         \rput(2.9,2.2){B}
         \rput(0.2,0.2){\cone}
         \rput(2.3,0.2){\boule}
         \rput(1.1,0.2){\cube} 
      \end{pspicture}
      \begin{pspicture}(0,0)(3.3,2.9)
         \psframe(0,0)(3.2,2.5)
         \rput(2.9,2.2){C}
         \rput(2.15,0.2){\boule}
         \rput(0.8,0.2){\cone}
         \rput(0.55,0.2){\cube}      
      \end{pspicture}
      \begin{pspicture}(0,0)(3.3,2.9)
         \psframe(0,0)(3.2,2.5)
         \rput(2.9,2.2){D}
         \rput(1.4,0.2){\cube}  
         \rput(0.4,0.2){\cone}
         \rput(0.85,0.2){\boule}
      \end{pspicture}
      \begin{pspicture}(0,0)(3,2.9)
         \psframe(0,0)(3.2,2.5)
         \rput(2.9,2.2){E}
         \rput(0.65,0.2){\cube}
         \rput(0.75,0.2){\boule}
         \rput(1.9,0.2){\cone}
      \end{pspicture} \par
      \begin{pspicture}(0,0)(3.3,2.7)
         \psframe(0,0)(3.2,2.5)
         \rput(2.9,2.2){F}
         \rput(1.6,0.2){\boule}
         \rput(1.3,0.2){\cubeg}
         \rput(0.5,0.2){\cone}
      \end{pspicture}
      \begin{pspicture}(0,0)(3.3,2.7)
         \psframe(0,0)(3.2,2.5)
         \rput(2.9,2.2){G}
         \rput(0.3,0.2){\cone}
         \rput(2.55,0.2){\boule}
         \rput(0.8,0.2){\cubeg} 
      \end{pspicture}
      \begin{pspicture}(0,0)(3.3,2.7)
         \psframe(0,0)(3.2,2.5)
         \rput(2.9,2.2){H}
         \rput(0.6,0.2){\cone}
         \rput(1.4,0.2){\cubeg}
         \rput(1.6,0.2){\boule}   
      \end{pspicture}
      \begin{pspicture}(0,0)(3.3,2.7)
         \psframe(0,0)(3.2,2.5)
         \rput(2.9,2.2){I}
         \rput(1.6,0.2){\cone}
         \rput(0.6,0.2){\cubeg}  
         \rput(1.5,0.2){\boule}
      \end{pspicture}
      \begin{pspicture}(0,0)(3,2.7)
         \psframe(0,0)(3.2,2.5)
         \rput(2.9,2.2){J}
         \rput(1.2,0.2){\cubeg}
         \rput(0.65,0.2){\boule}
         \rput(2,0.2){\cone}
      \end{pspicture}
   \end{center}

\end{Maquette}


%%% Récré %%%
\begin{Maquette}[Cours]{Theme={Activité récréative},Couleur={IndianRed}}
    
   \ARtitre{La relation d'Euler}

      Leonhard Euler, né le 15 avril 1707 à Bâle (Suisse) et mort à 76 ans le 7 septembre 1783 à Saint-Pétersbourg (Empire russe), est un mathématicien et physicien suisse. \par \bigskip
      On a reproduit page suivante une représentation en perspective cavalière de neuf solides. Pour chacun de ces solides, effectuer les actions suivantes à l'aide du tableau ci-dessous :
      \begin{itemize}
         \item retrouver son nom dans le tableau (indiquer le numéro n) ;
         \item trouver le nombre de faces F ;
         \item trouver le nombre de sommets S ;
         \item trouver le nombre d'arrêtes A ;
         \item calculer la valeur de F + S $-$ A, que remarque-t-on ?
         \item dire si le solide est régulier, c'est-à-dire si toutes ses faces sont identiques et régulières, et tous les angles du solide sont identiques ;
         \item dire si le solide est convexe, c'est-à-dire s'il n'a pas de \og creux \fg{} ou de trou ;
         \item enfin, dire s'il s'agit d'un solide de Platon (polyèdre régulier et convexe).
      \end{itemize}
      \bigskip
      \begin{center}   
         {\hautab{2.5}
         \begin{tabular}{|c|*{4}{C{0.9}|}*{4}{C{1.5}|}}
            \hline
            Nom du solide & n & F & S & A & F + S $-$ A & est-il régulier ? & est-il convexe ? & solide de Platon ? \\
            \hline
            Tétraèdre & & & & & & & & \\
            \hline
            Polyèdre étoilé & & & & & & & & \\
            \hline
            Octaèdre & & & & & & & & \\
            \hline
            Pyramide & & & & & & & & \\
            \hline
            Icosaèdre & & & & & & & & \\
            \hline
            Prisme & & & & & & & & \\
            \hline
            Cube & & & & & & & & \\
            \hline
            Beignoïde  & & & & & & & & \\
            \hline
            Dodécaèdre & & & & & & & & \\
            \hline    
         \end{tabular}}
      \end{center}
   
   \vfill\hfill{\footnotesize\it Source : d'après une activité parue dans la revue {\og Envol \fg} n\degre129, octobre-novembre-décembre 2004.}
   
\pagebreak

   \ARtitre{Les solides}

      {\psset{unit=0.95}
      \begin{pspicture}(-4,-7)(14.7,19)
         \rput(12,16){\psSolid[object=cube,a=1,RotZ=20,action=draw*,fillcolor=DarkOrange!50] \par {\Huge2}}
         \rput(6,-4){\psSolid[object=tetrahedron,r=1.1,RotZ=20,action=draw*,fillcolor=DodgerBlue!50] \par {\Huge9}}
         \rput(12,0){\psSolid[object=dodecahedron,a=0.9,RotZ=0,action=draw*,fillcolor=Crimson!50] \par {\Huge8}}
         \rput(0,8){\psSolid[object=octahedron,a=1,RotZ=20,action=draw*,fillcolor=Yellow!50] \par {\Huge4}}
         \rput(6,4){\psSolid[object=icosahedron,a=1,RotZ=60,action=draw*,fillcolor=LightGrey!50] \par {\Huge6}}
         \rput(12,7.7){\psSolid[object=prisme,h=0.6,RotZ=10,action=draw*,fillcolor=ForestGreen!50] \par {\Huge5}}
         \rput(0,17){\psSolid[object=new, sommets=0 -0.7 0 -0.7 0 0 0 1 0 1 0 0 0 0 -2,faces={[3 2 1 0][4 0 3][4 3 2][4 2 1][4 1 0]},RotZ=40,action=draw*,fillcolor=DarkViolet!50]{\Huge 1}}
         \rput(5.5,12){\psSolid[object=anneau,r=0.4,R=1,h=1.5,ngrid=4,RotZ=30,RotY=50,action=draw*,fillcolor=Cyan!50] \par {\Huge 3}}
         \rput(0,0){\psSolid[object=new,sommets=-0.3 -0.3 -0.3 0.3 -0.3 -0.3 0.3 0.3 -0.3 -0.3 0.3 -0.3 -0.3 -0.3 0.3 0.3 -0.3 0.3 0.3 0.3 0.3 -0.3 0.3 0.3 1.5 0 0 0 1.5 0 0 0 1.5 0 0 -1.5 -1.5 0 0 0 -1.5 0,faces={[1 2 6 5][0 3 2 1][4 5 6 7][0 1 5 4][3 7 6 2][0 4 7 3][11 3 2][11 0 3][11 1 0][11 2 1][12 7 3][12 4 7][12 0 4][12 3 0][13 1 5][13 0 1][13 4 0][13 5 4][8 2 6][8 6 5][8 5 1][8 1 2][9 3 7][9 7 6][9 6 2][9 2 3][10 6 7][10 5 6][10 7 4][10 4 5]},RotZ=20,RotX=10,action=draw*,fillcolor=green!50] \par {\Huge7}}
      \end{pspicture}}

   \end{Maquette}