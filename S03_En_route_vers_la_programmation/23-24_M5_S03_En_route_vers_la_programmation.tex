\graphicspath{{../../S03_En_route_vers_la_programmation/Images/}}

\themeA
\chapter{En route vers la programmation}
\label{S03}


\programme%
   {\item Notions d'algorithme et de programme. 
    \item Notion de variable informatique.
    \item Déclenchement d'une action par un événement.
    \item Séquences d'instructions, boucles, instructions conditionnelles.}
   {\item Écrire, mettre au point (tester, corriger) et exécuter un programme en réponse à un problème donné.}

\vfill

\begin{debat}{Débat : la fourmi de Langton, que se passe-t-il ensuite ?}
   La {\bf fourmi de Langton}, du nom de son inventeur scientifique américain {\it Christopher Langton}, est un petit programme informatique inventé vers la fin des années 1980. \par
   Il consiste en un automate qui se déplace dans un quadrillage suivant des règles simples. Il modélise le fait qu'un ensemble de comportements élémentaires peut donner lieu à un comportement complexe.
   \tcblower
      \psset{unit=0.2}
      \begin{pspicture}(0,0)(13,13)
         \fourmi{6.5}{6.5}{-90}
         \rput(2,0){\cub} \rput(3,0){\cub}
         \rput(1,1){\cub} \rput(2,1){\cub} \rput(9,1){\cub} \rput(10,1){\cub}
         \rput(0,2){\cub} \rput(2,2){\cub} \rput(3,2){\cub} \rput(5,2){\cub} \rput(8,2){\cub} \rput(11,2){\cub}
         \rput(0,3){\cub} \rput(3,3){\cub} \rput(5,3){\cub} \rput(6,3){\cub} \rput(7,3){\cub} \rput(9,3){\cub} \rput(10,3){\cub} \rput(11,3){\cub}
         \rput(1,4){\cub} \rput(3,4){\cub} \rput(10,4){\cub} \rput(12,4){\cub}
         \rput(3,5){\cub} \rput(4,5){\cub} \rput(8,5){\cub} \rput(9,5){\cub}
         \rput(3,6){\cub} \rput(4,6){\cub} \rput(5,6){\cub} \rput(7,6){\cub} \rput(8,6){\cub} \rput(9,6){\cub}
         \rput(3,7){\cub} \rput(4,7){\cub} \rput(8,7){\cub} \rput(9,7){\cub}
         \rput(0,8){\cub} \rput(2,8){\cub} \rput(9,8){\cub} \rput(11,8){\cub}
         \rput(0,9){\cub} \rput(1,9){\cub} \rput(2,9){\cub} \rput(5,9){\cub} \rput(6,9){\cub} \rput(7,9){\cub} \rput(9,9){\cub} \rput(12,9){\cub}
         \rput(1,10){\cub} \rput(4,10){\cub} \rput(7,10){\cub} \rput(9,10){\cub} \rput(10,10){\cub} \rput(12,10){\cub}
         \rput(2,11){\cub} \rput(3,11){\cub} \rput(10,11){\cub} \rput(11,11){\cub}
         \rput(9,12){\cub} \rput(10,12){\cub}
      \end{pspicture}
\end{debat}

\hfill {\gray Vidéo : \href{https://www.youtube.com/watch?v=qZRYGxF6D3w}{\bf La fourmi de Langton}, chaîne YouTube {\it Science étonnante}.}


%%% Approche %%%
\begin{Maquette}[Cours]{Theme={Activité d'approche},Couleur={SteelBlue}}

   \AAtitre{La fourmi de Langton}

      {\it Objectifs : suivre un algorithme de déplacement ; se repérer dans le plan dans un repérage relatif.}

      \begin{AActivite}

         \AApartie{Les règles du jeu}
            La fourmi de Langton est un automate qui se déplace dans un quadrillage suivant les règles suivantes :
            {\singlespacing
            \begin{itemize}
               \item au départ, toutes les cases sont de la même couleur, ici blanches ;
               \item si la fourmi est sur une case blanche, elle tourne de 90° vers la droite, change la couleur de la case en noir et avance d'une case ;
               \item si la fourmi est sur une case noire, elle tourne de 90° vers la gauche, change la couleur de la case en blanc et avance d'une case.
            \end{itemize}}

         \AApartie{À vous de jouer !}
            Compléter dans les quadrillages ci-dessous les quinze premières étapes du déplacement de la fourni. \par
            \begin{center}
               \psset{unit=0.5,subgriddiv=1,gridlabels=0mm,gridcolor=gray}
               \small
               \begin{pspicture}(0,-0.7)(7,7)
                  \psgrid(0,0)(7,7)
                  \fourmi{3.5}{3.5}{0}
                  \rput(3.5,-0.5){étape 0}
               \end{pspicture}
               \quad
               \begin{pspicture}(0,-0.7)(7,7)
                  \psframe[fillstyle=solid,fillcolor=darkgray](3,3)(4,4)
                  \psgrid(0,0)(7,7)
                  \fourmi{4.5}{3.5}{-90}
                  \rput(3.5,-0.5){étape 1}
               \end{pspicture}
               \quad
               \begin{pspicture}(0,-0.7)(7,7)  
                  \psframe[fillstyle=solid,fillcolor=darkgray](3,3)(5,4)
                  \psgrid(0,0)(7,7)
                  \fourmi{4.5}{2.5}{180}
                  \rput(3.5,-0.5){étape 2}
               \end{pspicture}
               \quad
               \begin{pspicture}(0,-0.7)(7,7)
                  \psgrid(0,0)(7,7)
                  \rput(3.5,-0.5){étape 3}
               \end{pspicture}
               \bigskip
               \begin{pspicture}(0,-0.7)(7,7)
                  \psframe[fillstyle=solid,fillcolor=darkgray](3,2)(5,4)
                  \psgrid(0,0)(7,7)
                  \fourmi{3.5}{3.5}{0}
                  \rput(3.5,-0.5){étape 4}
               \end{pspicture}
               \quad
               \begin{pspicture}(0,-0.7)(7,7)
                  \psgrid(0,0)(7,7)
                  \rput(3.5,-0.5){étape 5}
               \end{pspicture}
               \quad
               \begin{pspicture}(0,-0.7)(7,7)
                  \psgrid(0,0)(7,7)
                  \rput(3.5,-0.5){étape 6}
               \end{pspicture}
               \quad
               \begin{pspicture}(0,-0.7)(7,7)
                  \psframe[fillstyle=solid,fillcolor=darkgray](2,3)(3,5)
                  \pspolygon[fillstyle=solid,fillcolor=darkgray](3,2)(5,2)(5,4)(4,4)(4,3)(3,3)
                  \psgrid(0,0)(7,7)
                  \fourmi{3.5}{4.5}{-90}
                  \rput(3.5,-0.5){étape 7}
               \end{pspicture}
               \bigskip
               \begin{pspicture}(0,-0.7)(7,7)
                  \psgrid(0,0)(7,7)
                  \rput(3.5,-0.5){étape 8}
               \end{pspicture}
               \quad
               \begin{pspicture}(0,-0.7)(7,7)
                  \psgrid(0,0)(7,7)
                  \rput(3.5,-0.5){étape 9}
               \end{pspicture}
               \quad
               \begin{pspicture}(0,-0.7)(7,7)
                  \psgrid(0,0)(7,7)
                  \rput(3.5,-0.5){étape 10}
               \end{pspicture}
               \quad
               \begin{pspicture}(0,-0.7)(7,7)
                  \pspolygon[fillstyle=solid,fillcolor=darkgray](2,2)(5,2)(5,4)(4,4)(4,5)(2,5)(2,4)(3,4)(3,3)(2,3)
                  \psgrid(0,0)(7,7)
                  \rput(3.5,-0.5){étape 11}
                  \fourmi{1.5}{2.5}{90}
               \end{pspicture}
               \bigskip
               \begin{pspicture}(0,0)(7,7)
                  \psgrid(0,0)(7,7)
                  \rput(3.5,-0.5){étape 12}
               \end{pspicture}
               \quad
               \begin{pspicture}(0,0)(7,7)
                  \psgrid(0,0)(7,7)
                  \rput(3.5,-0.5){étape 13}
               \end{pspicture}
               \quad
               \begin{pspicture}(0,0)(7,7)
                  \psgrid(0,0)(7,7)
                  \rput(3.5,-0.5){étape 14}
               \end{pspicture}
               \quad
               \begin{pspicture}(0,0)(7,7)
                  \psgrid(0,0)(7,7)
                  \rput(3.5,-0.5){étape 15}
               \end{pspicture} 
            \end{center}

      \end{AActivite}

\end{Maquette}


%%% Trace écrite %%%
\begin{Maquette}[Cours]{Theme={Trace écrite},Couleur={0.4[SteelBlue,Black]}}

   %%% 1
   \section{Algorithmes et langages de programmation}

      \begin{definition*}{}
         Un {\bf algorithme} est une liste ordonnée et logique d'instructions permettant de réaliser une tâche de manière automatisée.
      \end{definition*}
   
      Un algorithme peut-être traduit, grâce à un langage de programmation, en un programme exécutable par un ordinateur. Ce langage peut être formel, textuel, visuel\dots{} Actuellement, le logiciel utilisé au collège est Scratch, développé par le MIT. Les programmes sont créés grâce à une succession de blocs, chacun ayant une fonction.
      \begin{center}
         \includegraphics[width=17cm]{Interface_Scratch}
      \end{center}
   
   
   %%% 2
   \section{Se déplacer}
   
      \begin{methode*}{Langages de déplacement}
         Pour se déplacer dans le plan, il existe principalement deux langages de déplacement :
         \begin{itemize}
            \item le langage {\bf absolu} composé de mots de vocabulaire du type : \og haut \fg{}, \og bas \fg{}, \og droite \fg{} et \og gauche \fg. Le déplacement se fait comme si on se plaçait en vue du dessus ;
            \item le langage {\bf relatif} composé de mots de vocabulaire du type : \og avancer \fg{}, \og tourner à droite \fg{} et \og tourner à gauche \fg. C'est ici le point de vue de l'observateur qui est adopté.
         \end{itemize}
         \begin{exmethode}
            Coder ce déplacement :
            \begin{center}
               \psset{unit=0.7}
               \begin{pspicture}(0,-1)(5,4)
                  \psgrid[subgriddiv=1,gridlabels=0mm](0,-1)(5,4)
                  \psset{linecolor=DodgerBlue,arrowsize=3mm,linewidth=0.5mm}
                  \psdots(0.5,0.5)(3.5,2.5)     
                  \psline{->}(0.5,0.5)(2.5,0.5)(2.5,2.5)(3.5,2.5)
               \end{pspicture}
            \end{center}
            \tcblower
               \begin{minipage}{3.3cm}
                  {\bf Langage absolu :} \par
                  droite \par
                  droite \par
                  haut \par
                  haut \par
                  droite
               \end{minipage}
               \;
               \begin{minipage}{3.3cm}   
                  {\bf Langage relatif :} \par
                  avancer \par
                  avancer \par
                  tourner à gauche \par
                  avancer \par
                  avancer \par
                  tourner à droite \par
                  avancer
               \end{minipage}
         \end{exmethode}
      \end{methode*}

\end{Maquette}


%%% Exercices %%%%%%%
\begin{Maquette}[Fiche,CorrigeFin,Colonnes=2]{}
   
   \begin{exercice} %1
      \begin{minipage}{8cm}
         Un programme permet à un robot de se déplacer sur les cases d'un quadrillage. \par
         Chaque case atteinte est colorée en gris. Au début d'un programme, toutes les cases sont blanches, le robot se positionne sur une case de départ indiquée par un \og {\bf d} \fg{} et la colore aussitôt en gris. \par
         Le robot se déplace suivant un programme grâce à un langage absolu dont le vocabulaire est
         \begin{center}
            \og S (south) ; E (east) ; N (north) ; W (west) \fg.
         \end{center}
         On a ci-contre des exemples de programmes.
      \end{minipage}
      \qquad
      \begin{minipage}{8cm}
         \begin{tabular}{|p{1.5cm}|C{2.5}|C{2.6}|}
            \hline
            1W
            &
            Le robot avance de 1 case vers l'ouest.
            &
            {\psset{unit=0.5cm}
            \begin{pspicture}(1,0)(3,3)
               \psframe[fillstyle=solid,fillcolor=lightgray](1,1)(3,2)
               \psgrid[gridlabels=0,subgriddiv=1,gridcolor=gray](0,0)(4,3)
               \rput(2.5,1.5){\textbf{d}}
            \end{pspicture}} \\
            \hline
            2E 1W 2N
            &
            Le robot avance de 2 cases vers l'est, puis de 1 case vers l'ouest, puis de 2 cases vers le nord.
            &
            {\psset{unit=0.5cm}
            \begin{pspicture}(0,4)(5,5)
               \pspolygon[fillstyle=solid,fillcolor=lightgray](1,1)(4,1)(4,2)(3,2)(3,4)(2,4)(2,2)(1,2)
               \psgrid[gridlabels=0,subgriddiv=1,gridcolor=gray](5,5)
               \rput(1.5,1.5){\textbf{d}}
            \end{pspicture}} \\
            \hline
         \end{tabular}
      \end{minipage}
      \begin{enumerate}
         \item Voici un programme : 1W 2N 2E 4S 2W \par
            On souhaite dessiner le motif obtenu avec ce programme. Sur votre copie, réaliser ce motif en utilisant des carreaux, comme dans les exemples précédents. On marquera un \og \textbf{d} \fg{} sur la case de départ.
         \item On fait fonctionner un programme qui dessine le motif ci-contre : \par
            \begin{minipage}{10cm}
               \begin{enumerate}
                  \item Proposer un programme permettant de dessiner ce motif.
                  \item Comment pourrait-on faire évoluer l'écriture de ce programme afin qu'il soit plus compact ?
               \end{enumerate}
            \end{minipage}
            \qquad
            \begin{minipage}{5cm}
               {\psset{unit=0.5cm}
               \begin{pspicture}(-1,-1)(8,2.5)
                  \pspolygon[fillstyle=solid,fillcolor=lightgray](0,2)(1,2)(1,1)(2,1)(2,2)(3,2)(3,1)(4,1)(4,2)(5,2)(5,1)(6,1)(6,2)(7,2)(7,0)(0,0)
                  \psgrid[gridlabels=0,subgriddiv=1,gridcolor=gray](-1,-1)(8,3)
                  \rput(0.5,1.5){\textbf{d}}
               \end{pspicture}}
            \end{minipage}
      \end{enumerate}
   \end{exercice}

   \begin{Solution}
      \begin{enumerate}
         \item On obtient le \cor{dessin d'un 9} : \par
            {\psset{linecolor=white,fillstyle=solid,fillcolor=white}
            \begin{pspicture}(-1,0)(4,7.3)
               \psframe[fillcolor=lightgray](1,1)(4,6)
               \psframe(1,2)(3,3)          
               \psframe(2,4)(3,5)
               \psgrid[gridlabels=0,subgriddiv=1,gridcolor=gray](5,7)
               \rput(1.5,1.5){\textbf{d}}
            \end{pspicture}}
         \item 
         \begin{enumerate}
            \item Le motif peut être programmé grâce à la suite : \par
               \cor{1S 2E 1N 1S 2E 1N 1S 2E 1N}
            \item On peut introduire une boucle de répétition, par exemple : \cor{3$\times$(1S 2E 1N)}
         \end{enumerate}
      \end{enumerate}
   \end{Solution}
   
   
   \begin{exercice}[Dur] %2
      Tracer les figures obtenues lorsque l'on exécute les programmes suivants avec scratch. Pour chaque cas, donner la nature de la figure obtenue. {\it On représentera l'unité (un pas )par 1 mm sur le cahier.} \par \smallskip
      \small Programme 1 \hspace*{2.05cm} Programme 2 \hspace*{2.05cm} Programme 3 \hspace*{2.05cm} Programme 4 \par \smallskip
      \begin{Scratch}[Echelle=0.75]
         Place Drapeau;
         Place PoserStylo;
         Place Repeter("4");
            Place Tournerd("90");
            Place Avancer("50");
         Place FinBlocRepeter;
      \end{Scratch}
      \quad
      \begin{Scratch}[Echelle=0.75]
         Place Drapeau;
         Place PoserStylo;
         Place Repeter("3");
            Place Avancer("40");
            Place Tournerg("120");
         Place FinBlocRepeter;
      \end{Scratch}
      \quad
      \begin{Scratch}[Echelle=0.75]
         Place Drapeau;
         Place PoserStylo;
         Place Repeter("2");
            Place Avancer("20");
            Place Tournerd("90");
            Place Avancer("80");
            Place Tournerd("90");
         Place FinBlocRepeter;
      \end{Scratch} 
      \quad
      \begin{Scratch}[Echelle=0.75]
         Place Drapeau;
         Place PoserStylo;
         Place Repeter("2");
            Place Avancer("30");
            Place Tournerd("50");
            Place Avancer("30");
            Place Tournerd("130");
         Place FinBlocRepeter;
      \end{Scratch}
   \end{exercice}

   \begin{Solution}
      Le programme 1 donne un \cor{carré} de côté \Lg{5}, le programme 2 un \cor{triangle équilatéral} de côté \Lg{4}, le programme 3 un \cor{losange} de côté \Lg{3} et le programme 4 un \cor{rectangle} de longueur \Lg{8} et de largeur \Lg{2}. \par
      {\psset{linecolor=blue}
      \begin{pspicture}(0,0)(7,7.3)                                                                              
         \psgrid[gridlabels=0,subgriddiv=0,gridcolor=lightgray](0,0)(7,7)
         \psdot[linewidth=0.7mm](1,6)
         \psframe(1,1)(6,6)    
      \end{pspicture}} \par   
      {\psset{linecolor=RoyalBlue}
      \begin{pspicture}(0,0)(8,9)                                                                              
         \psgrid[gridlabels=0,subgriddiv=0,gridcolor=lightgray](0,0)(8,9)    
         \psdot[linewidth=0.7mm](0,4)
         \pspolygon(0,4)(4,4)(2,7.46)
         \psdot[linewidth=0.7mm](6,8)
         \psframe(6,0)(8,8)
         \psdot[linewidth=0.7mm](0,3)
         \pspolygon(0,3)(3,3)(4.93,0.7)(1.93,0.7)
      \end{pspicture}}
   \end{Solution}
      
   \begin{multicols}{2}

      \begin{exercice} %3
         En utilisant les instructions ci-dessous, écrire un programme permettant de tracer les pointillés. \par
         {\psset{unit=0.8}
         \begin{pspicture}(0,-0.5)(9,0.5)
            \psset{linewidth=1mm,linecolor=RoyalBlue}
            \psline(2,0)(3,0)
            \psline(4,0)(5,0)
            \psline(6,0)(7,0)
            \psline(8,0)(9,0)
         \end{pspicture}} \par
         \begin{Scratch}[Echelle=0.7]
            Place PoserStylo;
            Place LigneVide;
            Place Avancer("10");
            Place LigneVide;
            Place ReleverStylo;
         \end{Scratch} 
         \qquad
         \begin{Scratch}[Echelle=0.7]
            Place Drapeau;
            Place LigneVide;
            Place Repeter("");
               Place LigneVide;
            Place FinBlocRepeter;      
         \end{Scratch}
      \end{exercice}

      \begin{Solution}
         On peut proposer le programme suivant : \par \smallskip
         \begin{Scratch}[Echelle=0.7]
            Place Drapeau;
            Place Repeter("5");
               Place PoserStylo;
               Place Avancer("10");
               Place ReleverStylo;
               Place Avancer("10");
            Place FinBlocRepeter;      
         \end{Scratch}
      \end{Solution}
      

      \begin{exercice}[Dur] %4
         Proposer un programme permettant de dessiner les marches d'un escalier comme ci-dessous. \par
         Chaque segment de la marche doit mesurer 100 pas. \par
         {\psset{unit=0.5}
         \begin{pspicture}(-4,0)(6,6.5)
            \psline[linewidth=1mm,linecolor=blue](0,6)(0,4)(2,4)(2,2)(4,2)(4,0)(6,0)
         \end{pspicture}}
      \end{exercice}
      
      \begin{Solution}
         On peut proposer le programme suivant : \par \smallskip
         \begin{Scratch}[Echelle=0.7]
            Place Drapeau;
            Place PoserStylo;
            Place Repeter("3");     
               Place Tournerd("90");
               Place Avancer("100");
               Place Tournerg("90");
               Place Avancer("100");
            Place FinBlocRepeter;      
         \end{Scratch}
      \end{Solution}
            
   \end{multicols}

\end{Maquette}


%%% Récré %%%
\begin{Maquette}[Cours]{Theme={Activité récréative},Couleur={IndianRed}}
    
   \ARtitre{Le jeu des dominogrammes}

      \ARpartie{But du jeu}
         En groupe, faire une chaîne fermée avec les huit cartes de domino.

      \ARpartie{Règle du jeu}
         Chaque domino est basé sur {\it Les douze travaux d'Hercule}, et notamment le travail n\degre11 dans lequel Hercule doit dérober les pommes d'or du jardin d'Hespérides. Le côté gauche comporte un quadrillage avec des cases noirs que l'on ne peut pas traverser, le personnage d'Hercule (orienté) et le pommier du jardin d'Hespérides. Le côté droit comporte un programme de déplacement d'Hercule. L'objectif est d'associer un programme d'un domino avec un quadrillage d'un autre domino. Les huit dominos doivent créer une chaîne fermée.

      \ARpartie{Jeu niveau 1}
         \begin{center}
            \psset{unit=0.4}
            \begin{pspicture}(-1,-1)(19,10) % jaune 1
               \psframe(-1,-1)(19,9)
               \psline(9,-1)(9,9)
               \psgrid[subgriddiv=1,gridlabels=0](0,0)(8,8)
               \put(3.1,4.1){\ho} \put(7.1,6.1){\po}
               \put(1,1){\cn} \put(2,3){\cn} \put(3,2){\cn} \put(2,4){\cn}  \put(5,5){\cn} \put(7,2){\cn} \put(5,7){\cn} \put(6,0){\cn} \put(1,6){\cn}     
               \put(10,7){\dep}
               \put(10,6){\av{3}}
               \put(10,5){\td}
               \put(10,4){\av{1}}
               \put(10,3){\tg}
               \put(10,2){\av{1}}
               \put(10,1){\fin}
               \put(18,8){\ding{40}}
            \end{pspicture}
            \qquad
            \begin{pspicture}(-1,-1)(19,9) % jaune 7
               \psframe(-1,-1)(19,9)
               \psline(9,-1)(9,9)
               \psgrid[subgriddiv=1,gridlabels=0](0,0)(8,8)
               \put(7.1,7.1){\reflectbox{\ho}} \put(5.1,6.1){\po}
               \put(6,6){\cn} \put(2,3){\cn} \put(0,6){\cn} \put(4,2){\cn}  \put(3,7){\cn} \put(1,2){\cn} \put(0,3){\cn} \put(7,2){\cn} \put(2,5){\cn}
               \put(10,7){\dep}
               \put(10,6){\av{2}}
               \put(10,5){\tg}
               \put(10,4){\tg}
               \put(10,3){\tg}
               \put(10,2){\tg}
               \put(10,1){\av{2}}
               \put(10,0){\fin}
               \put(18,8){\ding{87}}
            \end{pspicture}
         
            \medskip
            \begin{pspicture}(-1,-1)(19,9) % jaune 6
               \psframe(-1,-1)(19,9)
               \psline(9,-1)(9,9)
               \psgrid[subgriddiv=1,gridlabels=0](0,0)(8,8)
               \rput{-90}(4.5,7.5){\ho} \put(1.1,3.1){\po}
               \put(0,2){\cn} \put(6,5){\cn} \put(1,0){\cn} \put(2,0){\cn}  \put(3,6){\cn} \put(1,6){\cn} \put(6,7){\cn} \put(6,2){\cn} \put(1,7){\cn}     
               \put(10,7){\dep}
               \put(10,6){\av{2}}
               \put(10,5){\tg}
               \put(10,4){\av{1}}
               \put(10,3){\fin}
               \put(18,8){\ding{74}}
            \end{pspicture}
            \qquad
            \begin{pspicture}(-1,-1)(19,9) % jaune 4
               \psframe(-1,-1)(19,9)
               \psline(9,-1)(9,9)
               \psgrid[subgriddiv=1,gridlabels=0](0,0)(8,8)
               \put(0.1,0.1){\ho} \put(0.1,7.1){\po}
               \put(5,3){\cn} \put(4,3){\cn} \put(3,6){\cn} \put(0,6){\cn}  \put(5,5){\cn} \put(2,1){\cn} \put(2,5){\cn} \put(6,2){\cn} \put(1,3){\cn}     
               \put(10,7){\dep}
               \put(10,6){\av{3}}
               \put(10,5){\tg}
               \put(10,4){\td}
               \put(10,3){\av{2}}
               \put(10,2){\tg}
               \put(10,1){\av{2}}
               \put(10,0){\fin}
               \put(18,8){\ding{52}}
            \end{pspicture}
         
            \medskip
            \begin{pspicture}(-1,-1)(19,10) % jaune 2
               \psframe(-1,-1)(19,9)
               \psline(9,-1)(9,9)
               \psgrid[subgriddiv=1,gridlabels=0](0,0)(8,8)
               \rput{90}(3.5,2.5){\ho} \put(4.1,6.1){\po}
               \put(1,2){\cn} \put(4,3){\cn} \put(3,0){\cn} \put(2,7){\cn}  \put(5,6){\cn} \put(1,2){\cn} \put(0,7){\cn} \put(6,2){\cn} \put(1,3){\cn}     
               \put(10,7){\dep}
               \put(10,6){\td}
               \put(10,5){\av{4}}
               \put(10,4){\tg}
               \put(10,3){\av{3}}
               \put(10,2){\tg}
               \put(10,1){\av{1}}
               \put(10,0){\fin}
               \put(18,8){\ding{110}}
            \end{pspicture}
            \qquad
            \begin{pspicture}(-1,-1)(19,9) % jaune 5
               \psframe(-1,-1)(19,9)
               \psline(9,-1)(9,9)
               \psgrid[subgriddiv=1,gridlabels=0](0,0)(8,8)
               \rput{90}(5.5,0.5){\ho} \put(3.1,5.1){\po}
               \put(1,1){\cn} \put(0,3){\cn} \put(4,2){\cn} \put(3,4){\cn}  \put(5,6){\cn} \put(7,3){\cn} \put(5,7){\cn} \put(6,1){\cn} \put(1,7){\cn}     
               \put(10,7){\dep}
               \put(10,6){\av{6}}
               \put(10,5){\td}
               \put(10,4){\av{3}}
               \put(10,3){\td}
               \put(10,2){\av{2}}
               \put(10,1){\tg}
               \put(10,0){\fin}
               \put(18,8){\ding{70}}
            \end{pspicture}
         
            \medskip
            \begin{pspicture}(-1,-1)(19,9) % jaune 8
               \psframe(-1,-1)(19,9)
               \psline(9,-1)(9,9)
               \psgrid[subgriddiv=1,gridlabels=0](0,0)(8,8)
               \rput{90}(2.5,3.5){\ho} \put(2.1,7.1){\po}
               \put(5,3){\cn} \put(3,3){\cn} \put(3,6){\cn} \put(0,5){\cn}  \put(7,5){\cn} \put(2,1){\cn} \put(2,0){\cn} \put(6,2){\cn} \put(1,3){\cn}     
            \put(10,7){\dep}
               \put(10,6){\av{1}}
               \put(10,5){\tg}
               \put(10,4){\av{2}}
               \put(10,3){\td}
               \put(10,2){\av{2}}
               \put(10,1){\av{1}}
               \put(10,0){\fin}
               \put(18,8){\ding{115}}
            \end{pspicture}
            \qquad
            \begin{pspicture}(-1,-1)(19,9) % jaune 3
               \psframe(-1,-1)(19,9)
               \psline(9,-1)(9,9)
               \psgrid[subgriddiv=1,gridlabels=0](0,0)(8,8)
               \put(5.1,3.1){\reflectbox{\ho}} \put(2.1,6.1){\po}
               \put(6,6){\cn} \put(3,3){\cn} \put(3,6){\cn} \put(2,5){\cn}  \put(3,5){\cn} \put(4,2){\cn} \put(1,3){\cn} \put(0,0){\cn} \put(2,5){\cn}     
               \put(10,7){\dep}
               \put(10,6){\av{7}}
               \put(10,5){\tg}
               \put(10,4){\av{7}}
               \put(10,3){\tg}
               \put(10,2){\av{7}}
               \put(10,1){\fin}
               \put(18,8){\ding{57}}
            \end{pspicture}
         \end{center}

         Lorsque le groupe a réussi la mission, passer au niveau supérieur avec une autre série de dominos comportant des boucles de répétition. 

\end{Maquette}